\documentclass[12pt,a4paper, twosite]{article}

\usepackage[spanish]{babel}
\usepackage[utf8]{inputenc}
\usepackage[T1]{fontenc}
\usepackage{graphicx}
\usepackage{grffile}
\usepackage{longtable}
\usepackage{wrapfig}
\usepackage{rotating}
\usepackage[normalem]{ulem}
\usepackage{amsmath}
\usepackage{textcomp}
\usepackage{amssymb}
\usepackage{capt-of}
\usepackage{hyperref}
\usepackage[left=2.00cm, right=2.50cm, top=2.50cm, bottom=2.00cm]{geometry}
\usepackage{fancyhdr}
\fancyhead[RO,LE]{\thepage}
\usepackage{colortbl}
\usepackage{xcolor}
\definecolor{titDesc}{HTML}{1194D5}
\definecolor{rowMain}{HTML}{80A0E5}
\definecolor{row1}{HTML}{8FADEE}
\fancyhead[LO]{\emph{\uppercase{\leftmark}}}
\fancyfoot{}
\renewcommand{\headrulewidth}{1.0pt}
\pagestyle{fancy}
\date{}
\author{Gastón Valdez \\ gaston.cb.90@gmail.com}
\title{Telemetría y sistema de posicionamiento de antena para interferometría \\ Especificación de casos de uso}
\hypersetup{
	pdfborder={0 0 0},
	pdfauthor={Gastón Valdez},
	pdftitle={Reqerimientos de ingenieria de software},
	pdfkeywords={Requerimientos, posicionador de antena},
	pdfsubject={hola mundo},
	pdfcreator={Emacs 26.2 (Org mode 9.1.9)}, 
	pdflang={Spanish}
}


\begin{document}
	\begin{titlepage}
		\maketitle
	\end{titlepage}
\tableofcontents
	
	\newpage
	
	\section{Introducción}
	\label{sec:org60390fa}
	\subsection{Propósito}
	\label{sec:org434c3ef}
	El propósito del presente documento es definir los casos de uso posibles del sistema rotador que se han definido en los documentos \ref{ref:ptr},\ref{ref:ids} y \ref{ref:MIA}. 	
	
	\subsection{Definiciones, Acrónimos y Abreviaturas}
	\label{sec:orgb158e36}
	\begin{itemize}
	\item IAR: instituto Argentino de Radioastronomía.  
	\item SRS: Especifición de requerimientos de software.  
	\item EP: Electrónica de potencia.
	\item SBC: Single Board Computer.
	\item TBC: Falta de confirmación.
	\item TBD: Aún no definido.
	\item IdS: Ingeniería de software.
	\item N/A: No aplica.  
	\end{itemize}


	
	
	\subsection{Referencias}
	\label{sec:org62711e0}
	
\begin{enumerate}
	\item \label{ref:MIA}https://www.iar.unlp.edu.ar/slider/observatorio/
	\item \label{ref:cese} Plan de trabajo CESE.
	\item \label{ref:ptr} IAR-OBS-MIA-REQ-R05 (documento interno).
	\item \label{ref:ids} IdS\_Gaston\_Valdez.pdf (documento interno).

\end{enumerate}
	
	
	\subsection{Visión general del documento}
	\label{sec:orgdaca22c}
	Este documento se realiza siguiendo el estándar IEEE Std. 830-1998 de acuerdo con los lineamientos de la materia Ingeniería de Software de la carrera de especialización de sistema embebidos.

	
	\section{Descripción general }
	\label{sec:orgc1c4017}
	Descriptas en \ref{ref:ids}.
	\subsection{Perspectiva del producto}
	\label{sec:org24980a8}
	Descriptas en \ref{ref:ids}.

	\subsection{Funciones del producto}
	\label{sec:orgaf51da6}
	Descriptas en \ref{ref:ids}.
	
	\section{Casos de uso}
	\label{sec:org40573d1}
		
	\subsection{caso de uso 1 }
	%1989C2 cyan 
	\begin{tabular}{|p{0.001\textwidth}|p{0.25\textwidth}|p{0.2\textwidth}|p{0.5\textwidth}|}
		\hline 
		\rowcolor{row1}  \multicolumn{2}{|c|}{Título} & \multicolumn{2}{|c|}{Descripción} \\ \hline
		\rowcolor{titDesc}
		\multicolumn{2}{|c|}{1. Nombre} & \multicolumn{2}{|c|}{Seguimiento de radiofuentes} \\ \hline
		
		& 1.1. Breve descripción & \multicolumn{2}{p{0.7\textwidth}|}{En este escenario se configura una radiofuente a seguir}  \\ \cline{2-4} 
		& 1.2. Actor Principal & \multicolumn{2}{p{0.7\textwidth}|}{Operadores de antena}  \\ \cline{2-4}  
		&1.3. Disparadores & \multicolumn{2}{p{0.7\textwidth}|}{Estrella o satélite a seguir. El seguimiento es automático, puede ser mediante una agenda o manual. }  \\ \hline 
		
		%consultar si unir las dos filas ! 
		%\columncolor{rowMain}
		%\rowcolor{rowMain}
		\multicolumn{4}{|>{\columncolor{rowMain}[8pt][19pt]}p{\textwidth}|}{2. Flujo de eventos} \\ \hline 
		&2.1. Flujo básico &\multicolumn{2}{p{0.7\textwidth}|}{2.1.1. Elegir programa de apuntamiento (Gpredict,Stellarium o scripts bash)\newline 2.1.2. Selección de la fuente a seguir  \newline 2.1.3. El dispositivo recibe las coordenadas y debe realizar el cálculo del seguimiento\newline 2.1.4. Estaciona la antena
		esperando la salida del astro/satelite 
		\newline 2.1.5. Realiza el seguimiento durante toda la trayectoria del astro/satelite, hasta que vuelva a esconderse por la linea del horizonte.  
		\newline 2.1.5. Vuelve a la posición del cenit }  \\ \cline{2-4}  
		
		% consultar si hay que unir las filas 
		& 2.2. Flujo alternativo   &  \multicolumn{2}{p{0.7\textwidth}|}{
			2.2.1. El sistema detecta que las coordenadas son erroneas \newline2.2.2. Alerta al operador de la situación }  \\ \hline  
		
		
		\multicolumn{2}{|p{0.3\textwidth}|}{3. Requerimientos especiales } & 
		\multicolumn{2}{|p{0.7\textwidth}|}{ 3.1. El viento no supera los 50 km/h durante toda la operación de seguimiento}  \\ \hline  
		\multicolumn{2}{|p{0.3\textwidth}|}{4. Pre-condiciones }  & \multicolumn{2}{p{0.7\textwidth}|}{ 4.1. Estado de UNTRACKING }  \\   
		\hline 
		\multicolumn{2}{|p{0.3\textwidth}|}{5. Post-condiciones } 
		&   \multicolumn{2}{p{0.7\textwidth}|}{5.1. Estado de UNTRACKING}  \\ \hline  
		
	\end{tabular}
	
	
	\subsection{caso de uso 2 }
	
	\begin{tabular}{|p{0.001\textwidth}|p{0.28\textwidth}|p{0.2\textwidth}|p{0.5\textwidth}|}
		\hline 
		%\rowcolor{red}
		%\multicolumn{4}{|p{\textwidth}|}{} \\ \hline una sola fila en primera columnta 
		\rowcolor{row1}	\multicolumn{2}{|c|}{Título} & \multicolumn{2}{|c|}{Descripción} \\ \hline
		\rowcolor{titDesc}	\multicolumn{2}{|c|}{1. Nombre} & \multicolumn{2}{|c|}{Vientos superiores a 50 km/h} \\ \hline
		
		& 1.1. Breve descripción & \multicolumn{2}{p{0.7\textwidth}|}{En este escenario se intenta seguir una radiofuente con un viento superior a 50 km/h}  \\ \cline{2-4} 
		& 1.2. Actor Principal & \multicolumn{2}{p{0.7\textwidth}|}{Operadores de antena}  \\ \cline{2-4}  
		&1.3. Disparadores & \multicolumn{2}{p{0.7\textwidth}|}{Estrella o satélite a seguir }  \\ \hline 
		
		%consultar si unir las dos filas ! 
		
		\multicolumn{4}{|>{\columncolor{rowMain}[7pt][22pt]}p{\textwidth}|}{2. Flujo de eventos} \\ \hline 
		&2.1. Flujo básico &\multicolumn{2}{p{0.7\textwidth}|}{2.1.1. Elegir programa de apuntamiento (Gpredict,Stellarium o scripts bash)\newline 2.1.2. Selección de la fuente a seguir  \newline 2.1.3. El dispositivo recibe las coordenadas e independiente de la precondición vuelve al cenit  \newline 2.1.4. Alerta al operador \newline 2.1.5. Vuelve a la posición del cenit  
			 }  \\ \cline{2-4}  
		
		% consultar si hay que unir las filas 
		& 2.2. Flujo alternativo   &  \multicolumn{2}{p{0.7\textwidth}|}{
			2.2.1. El sistema no puede volver al cenit \newline 
			2.2.2. Debe volver manualmente al cenit (sistema manual) }  \\ \hline  
		
		
		\multicolumn{2}{|p{0.3\textwidth}|}{3. Requerimientos especiales } & 
		\multicolumn{2}{p{0.7\textwidth}|}{ 3.1 El sistema debe estar encendido al menos 10 minutos.}  \\ \hline  
		\multicolumn{2}{|p{0.3\textwidth}|}{4 Pre-condiciones }  & \multicolumn{2}{p{0.7\textwidth}|}{ 4.1. Estado de UNTRACKING }  \\   
		\hline 
		\multicolumn{2}{|p{0.3\textwidth}|}{5. Post-condiciones } 
		&   \multicolumn{2}{p{0.7\textwidth}|}{5.1. Estado de CENIT}  \\ \hline  
		
	\end{tabular}
	
	
	\subsection{caso de uso 3 }
	
	
	\begin{tabular}{|p{0.001\textwidth}|p{0.28\textwidth}|p{0.2\textwidth}|p{0.5\textwidth}|}
		\hline 
		%\rowcolor{red}
		%\multicolumn{4}{|p{\textwidth}|}{} \\ \hline una sola fila en primera columnta 
		\rowcolor{row1}	\multicolumn{2}{|c|}{Título} & \multicolumn{2}{|c|}{Descripción} \\ \hline
		\rowcolor{titDesc}	\multicolumn{2}{|c}{1. Nombre} & \multicolumn{2}{|c|}{Al menos dos operadores con conexión a la antena } \\ \hline
		
		& 1.1. Breve descripción & \multicolumn{2}{p{0.7\textwidth}|}{Un operador intenta realizar un movimiento cuando la antena esta en estado de TRAKING}  \\ \cline{2-4} 
		& 1.2. Actor Principal & \multicolumn{2}{p{0.7\textwidth}|}{Operadores de antena}  \\ \cline{2-4}  
		&1.3. Disparadores & \multicolumn{2}{p{0.7\textwidth}|}{Estrella o satélite a seguir }  \\ \hline 
		%\rowcolor{rowMain}	
		\multicolumn{4}{|>{\columncolor{rowMain}[8pt][22pt]}p{\textwidth}|}{2. Flujo de eventos} \\ \hline 
		&2.1. Flujo básico &\multicolumn{2}{p{0.7\textwidth}|}{2.1.1. Detectar que operador solicitó primero el movimiento de la antena \newline 
			2.1.2. Cargar en una lista de espera y avisar al segundo operador de esta situación \newline	
			2.1.3. Mover la antena y realizar el seguimiento de este operador \newline
			2.1.4. Al desaperecer por el horizonte el primer astro/satelite, verificar si es posible satisfacer el segundo en base a los horarios de entrada y puesta por el horizonte.
			\newline
			2.1.5. Si es posible realizar el segundo seguimiento debe realizarlo.
			2.1.6. Si no es posible realizar el segundo seguimiento debe volver la antena al cenit}  \\ \cline{2-4}  
		% consultar si hay que unir las filas 
		& 2.2. Flujo alternativo   &  \multicolumn{2}{p{0.7\textwidth}|}{
			N/A
		}  \\ \hline  
		
		\multicolumn{2}{|p{0.3\textwidth}|}{3. Requerimientos especiales } & 
		\multicolumn{2}{p{0.7\textwidth}|}{N/A
		}  \\ \hline  
		\multicolumn{2}{|p{0.3\textwidth}|}{4 Pre-condiciones }  & \multicolumn{2}{p{0.7\textwidth}|}{N/A }  \\   
		\hline 
		\multicolumn{2}{|p{0.3\textwidth}|}{5. Post-condiciones } 
		&   \multicolumn{2}{p{0.7\textwidth}|}{5.1. Estado de UNTRACKING o CENIT}  \\ \hline  
		
	\end{tabular}
	
	
	

\end{document}